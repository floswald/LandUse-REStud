

% \documentclass[12pt]{article}
\documentclass[11pt]{article}
% \usepackage[latin9]{inputenc}
\usepackage{geometry}
% \usepackage[left=0.7in,right=0.7in,top=0.7in,bottom=0.7in,nohead]{geometry} % extend the margin

\geometry{verbose,lmargin=2.4cm,rmargin=2.4cm,bmargin=2.5cm,tmargin=2.5cm}
\setlength{\parskip}{\smallskipamount}
\usepackage{parskip}  % no indent of first sentence in paragraph
\usepackage{color}
\usepackage{float}
\usepackage{url}
\usepackage{amsmath}
\usepackage{amsfonts}
\usepackage{amssymb}
\usepackage{amsthm}
\usepackage{hyperref}
\usepackage{booktabs}
\usepackage{dcolumn}
\usepackage{graphicx}
\usepackage{subcaption}

% \usepackage{subfig} old package, no longer to be used
\usepackage{natbib}
\usepackage{xcolor}
\usepackage{adjustbox}
\usepackage[normalem]{ulem}
\usepackage{longtable}
\usepackage{colortbl}
\usepackage{multirow}
\usepackage{hhline}
\usepackage{calc}
\usepackage{tabularx}
\usepackage{threeparttable}
\usepackage{wrapfig}
\usepackage{tikz}
\usepackage{xr}
\externaldocument{20250130-appendixA}
\externaldocument{20250130-appendixB}


\usepackage{setspace}
\onehalfspacing

%\usepackage{xcolor}
\definecolor{dark-red}{rgb}{0.8,0.15,0.15}
\definecolor{dark-blue}{rgb}{0.15,0.15,0.8}
\definecolor{medium-blue}{rgb}{0,0,0.5}
\hypersetup{
	colorlinks, linkcolor={dark-red},
	citecolor={dark-red}, urlcolor={dark-red}
}

\newtheorem{assumption}{Assumption}
\newtheorem{definition}{Definition}
\newtheorem{proposition}{Proposition}
\newtheorem{theorem}{Theorem}
\newtheorem{lemma}{Lemma}

% author affiliations in a footnote
% https://tex.stackexchange.com/a/214425
\newcommand{\footremember}[2]{%
	\footnote{#2}
	\newcounter{#1}
	\setcounter{#1}{\value{footnote}}%
}
\newcommand{\footrecall}[1]{%
	\footnotemark[\value{#1}]%
}

% revision version
% this needs to be set to the current revision 
\newcommand{\round}{revision3}  % 


% paths
\newcommand{\images}{../data/images}
\newcommand{\modplots}{../output/model/plots/\round/baseline}
\newcommand{\modtables}{../output/model/tables}
\newcommand{\dataplots}{../output/data/plots}
% \newcommand{\nicoplots}{../données_090620}
% \newcommand{\picsvids}{../../output/pics-vids}

\newcommand{\datatables}{../output/data/tables}
% \newcommand{\marcmatlab}{../../MatlabFiles_Marc/Spring2021}
\newcommand{\screenshots}{../data/images}
% \newcommand{\screenshots}{../../data/present-screenshots}


% commands
\providecommand{\huxb}[2]{\arrayrulecolor[RGB]{#1}\global\arrayrulewidth=#2pt}
\providecommand{\huxvb}[2]{\color[RGB]{#1}\vrule width #2pt}
\providecommand{\huxtpad}[1]{\rule{0pt}{\baselineskip+#1}}
\providecommand{\huxbpad}[1]{\rule[-#1]{0pt}{#1}}
\newcommand{\greenie}[1]{\textcolor[rgb]{0,0.71,0.35}{#1}}
\newcommand{\blueie}[1]{\textcolor[rgb]{0.1,0.0,0.95}{#1}}



% size of plot panels
\newcommand{\pthree}{0.37}  % proportional size of a panel in a 3-row plot
\newcommand{\ptwo}{0.5}  % proportional size of a panel in a 2-row plot

% shift the parbox of a 3x2 plot inside and adjustbox by this amount
\newcommand{\parboxadjust}{1.15}

%\usepackage{subfig}
\usepackage{float}
 \PassOptionsToPackage{caption=false}{subfig}


\usepackage{blindtext}
\usepackage{subfiles} % Best loaded last in the preamble

\begin{document}
	
	
\title{Structural Change, Land Use and Urban Expansion\thanks{We would like to thank Zs\'ofia B\'ar\'any, Thomas Chaney, Pierre-Philippe Combes, C\'ecile Gaubert, Emeric Henry and our colleagues as well as numerous seminar and conference participants for comments and insights. We thank Alberto Nasi and Carla Guerra Tomazini for research assistance. We gratefully acknowledge funding under the Banque de France-SciencesPo Partnership. Marc Teignier acknowledges financial support from the Spanish Ministry of Science, Innovation and Universities, grant PID2022-139468NB-I00, and AGAUR-Generalitat de Catalunya, grant 2021SGR00862, and BEAT. Numerical computations were partly performed on the S-CAPAD/DANTE platform, IPGP, France.}}
\author{%
	Nicolas Coeurdacier\\ \small SciencesPo Paris, CEPR
	\and Florian Oswald\\ \small Uni Turin ESOMAS,\\\small Collegio Carlo Alberto
	\and Marc Teignier\\ \small Serra Húnter Fellow, \\\small University of Barcelona
}

\maketitle

\begin{abstract}
	\noindent 
	 How do cities grow in the process of structural transformation? To answer this question, we develop a multi-sector spatial equilibrium model with endogenous land use: land is used either for agriculture or housing. Urban land, densely populated due to commuting frictions, expands out of agricultural land. With low productivity and high subsistence needs, farmland is expensive, households cannot afford large homes and cities are very dense. Increasing productivity reallocates factors away from agriculture, freeing up land for urban expansion and limiting the increase in land values despite higher income and urban population. With the area of cities growing faster than urban population, urban density can persistently decline, as in the data over a long period. The quantitative evaluation calibrated to historical data assembled for France over 180 years explains a large fraction of the joint evolution of urban areas, population density and land values across time and space. 
\end{abstract}

\textbf{Keywords}: Structural Change, Land Use, Productivity Growth, Urban Density.\\
\textbf{JEL-codes}: O41, R14, O11

\vfill

\pagebreak

\section{Introduction}
\subfile{20250130-sec-1-2}

\section{A Baseline Theory}
\label{sec:baselinemodel}
\subfile{20250130-section3}


\section{Quantitative evaluation for France (1840-2015)}\label{sec:QM} 
\subfile{20250130-section4}


\section{Conclusion}\label{sec:conclusion}

This paper develops a spatial general equilibrium model of structural change with endogenous land use and studies its implications for urbanization. We document a persistent fall of urban density in French cities since 1870 and show that the theoretical and quantitative predictions of the model are broadly consistent with the data. The quantitative version of the theory calibrated to French data explains about 70\% of the urban area expansion and most of the decline in average urban density, about half of the rise in housing prices, and most of the land value reallocation from rural to urban since the mid-nineteenth century. Novel predictions regarding urban density across space line up relatively well with available data.

Agricultural productivity growth is shown to be crucial for the results, since it reduces the price of land at the urban fringe and frees up resources to be spent on housing. As a consequence, while workers reallocate away from agriculture, cities grow faster in area than in population and land prices do not rise very rapidly. Faster commuting modes also play an important and complementary role but only when combined with rural growth and structural change. When rural productivity is high, they allow households to live further away from their workplace and enjoy larger homes, contributing significantly to the decline in urban density, particularly at the city center. 

Our baseline theory relies on a monocentric urban structure where all workers commute from their residential location to the center. While French cities exhibit the qualitative features of monocentric cities, such an urban structure certainly remains an approximation. Data show that commuting distance increases with residential distance to the center but less than one for one. This suggests that workers sort into jobs and residences that are closer to each other. Relaxing further the monocentric structure remains an important step to better account for the expansion of cities and the evolution of their density. We leave for future research a theory that jointly determines firms and workers location decisions across the urban space. More broadly, further heterogeneity across cities in their urban form seems necessary to account for the spatial dispersion of urban density.

Relatedly, we focus on the reallocation of economic activity from the rural to the urban sector, abstracting from the reallocation within the urban sector. We could extend our framework to consider the transition from manufactures to services in the later period. While aggregate results might not be much affected, we believe it would matter for the cross-section of cities in recent times. Some services are provided locally, especially in large cities, implying that not all workers have to commute to the center. We also leave this extension for future research. 

%We also believe that our approach can be used to study the aggregate implications of policies regulating land use and urban planning. Such policies are likely to play a role in explaining the evolution of housing prices in recent years, which our current setup cannot fully replicate. To the extent that land-use policies prevent cities from growing as much as they would, they lead to greater demand for available housing units and to a faster rise in their prices. The general equilibrium structure of our quantitative spatial model is well suited to conduct such policy counterfactuals.

We also believe that our approach can be used to study the aggregate implications of policies regulating land use and urban planning. Such policies are likely to play a role in explaining the evolution of housing prices in recent years, which our current setup cannot fully replicate. To the extent that land-use policies reduce city growth on the extensive margin, they lead to greater demand for available housing units and to faster rise in their prices. The general equilibrium structure of our quantitative spatial model is well suited to conduct such policy counterfactuals. More broadly, our framework can be used to revisit a variety of normative questions in presence of externalities. While our approach is positive, urban density is at the heart of agglomeration and congestion externalities on productivity. Depending on the context, population density or urban sprawl are also sources of pollution and environmental externalities. By bringing novel insights on the determinants of urban sprawl and urban density across time and space, our approach might shed novel light on the design of policies to address such externalities.

\pagebreak
\bibliography{../refs}
\bibliographystyle{aer}

\end{document} 

