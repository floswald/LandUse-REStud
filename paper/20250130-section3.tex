\documentclass[20250130-paper.tex]{subfiles}


\begin{document}




We present the baseline spatial equilibrium model, describing the environment, deriving equilibrium conditions and defining the equilibrium formally. %The model is extended in the next section in order to get quantitative predictions to confront with the data presented in the previous section.

\subsection{Environment Description}
\label{subsec:environment}
We consider an economy producing an urban good ($u$) and a rural good ($r$) at a given date. Time subscripts are omitted for convenience. The urban good is thought of as a composite of manufacturing goods and services, while the rural good represents the agricultural good. The urban good is also used in the production of housing services. Goods and factor markets are perfectly competitive. Both goods are perfectly tradable. The economy is composed of $K$ different regions indexed by $k \in \{1,..., K\}$ with different productivities. Labor is perfectly mobile across sectors and regions. 

\textbf{Factor Endowments.} The economy is endowed with land and a continuum of ex-ante identical workers, both in fixed supply. Each worker is endowed with one unit of labor and we denote by $L$ the total population of workers. Each region $k$ is endowed with land of area $S$. Land can be used to produce the rural good or for residential purposes. The production of the urban good takes place in the city of each region $k$, denoted city $k$, while the production of the rural good, being more land intensive, takes place in the rural area of the region. We assume that production of the urban good takes place in only one location in each city, namely location $\ell_k=0$ of city $k$, which is similar to the Central Business District (CBD) in a standard urban model. Regions are assumed to be circular of radius $\sqrt{S/\pi}$ and city $k$ is located at the center of its respective region. Workers' residence $\ell_k$ can lie anywhere in the region and is denoted by its distance $\ell_k$ from the center of city $k$ due to symmetry. 

\textbf{Technology.} The production of the urban good only uses labor as input. In each region $k$, one unit of labor produces $\theta_{u,k}$ units of the urban good
\begin{equation*}
Y_{u,k} =\theta_{u,k}L_{u,k}
\end{equation*}
where $L_{u,k}$ denotes the number of workers in the urban sector of region $k$.

In each region $k$, the production of the rural good uses labor and land according to the following constant returns to scale technology,
\begin{equation*}
Y_{r,k}=\theta_{r,k} (L_{r,k})^{\alpha}(S_{r,k})^{1-\alpha},
\end{equation*}
where $L_{r,k}$ denotes the number of workers working in the rural sector in region $k$, $S_{r,k}$ the amount of land used for production and $\theta_{r,k}$ a Hicks-neutral productivity parameter. $0<\alpha<1$ is the intensity of labor use in production. %$\sigma \geq 0$ is the elasticity of substitution between labor and land, $\sigma=1$ corresponding to the usual Cobb-Douglas case.

\textit{Remark.} The important technology assumption is that the rural sector is more land intensive than the urban one,  $1-\alpha>0$, implying stronger decreasing returns to scale to labor in this sector. %The fact that the urban sector does not use land is not crucial as long as this sector is less land intensive than the rural one. 

The production of housing space provided by land developers can use more or less intensively the land for residential purposes. In each location $\ell_k$ of region $k$, developers supply housing space $H(\ell_k)$ per unit of land with a convex cost, $\frac{H(\ell_k)^{1+1/\epsilon}}{1+1/\epsilon}$ with $\epsilon>0$, in units of the numeraire urban good.\footnote{The urban good is used as an intermediary input for the production of housing space. Some equivalent formulation holds for a Cobb-Douglas production function of housing (see \cite{combes2018costs}).} 

\textbf{Preferences.} Preferences over urban and rural goods are non-homothetic as in \cite{kongsamut} and \cite{herrendorf2013aer} among others. Consider a worker living in a location $\ell_k$ of region $k$. Denote $c_r(\ell_k)$ the consumption of rural goods, $c_u(\ell_k)$ the consumption of urban goods (used as numeraire) and $h(\ell_k)$ the consumption of
housing. Workers derive utility only from consumption in location $\ell_k$, which is defined as
\begin{equation}
C(\ell_k)=\mathcal{C}\left(c_{r}(\ell_k), c_{u}(\ell_k) \right)^{1-\gamma}h(\ell_k)^\gamma, \label{eq:U}
\end{equation}
where the housing preference parameter $\gamma$ belongs to $(0,1)$ and the consumption composite $\mathcal{C}$ over rural and urban goods is a non-homothetic CES aggregate with substitution elasticity $\sigma$,
\begin{equation*}
\mathcal{C}\left(c_{r}(\ell), c_{u}(\ell)\right) =\left[\nu^{1/\sigma}\left(c_{r}(\ell)-\underline{c}\right)^{\frac{\sigma-1}{\sigma}}+(1-\nu)^{1/\sigma}\left(c_{u}(\ell)+s\right)^{\frac{\sigma-1}{\sigma}}\right]^{\frac{\sigma}{\sigma-1}}. \label{eq:C}
\end{equation*}
$\underline{c}$ denotes the minimum consumption level for the rural (subsistence) good, $\underline{s}$ stands for the initial endowment of the urban (luxury) good and the preference parameter $\nu$ belongs to $(0,1)$. Preferences are Stone-Geary for $\sigma=1$. 

\textbf{Urban Spatial Structure.} Workers face spatial frictions $\tau(\ell_k)$ when commuting to work in the urban sector of city $k$. A worker residing in location $\ell_k$ and working in the urban sector earns a wage \textit{net of spatial frictions} equal to $w(\ell_k)=w_{u,k} - \tau(\ell_k)$, with $w_{u,k}$ denoting the urban wage in city $k$, $\tau(0)=0$, and $\partial \tau(\ell_k)/\partial \ell_k \geq 0$. The commuting cost $\tau(\ell_k)$ incorporates all spatial frictions which lower disposable income available for consumption when living further away from the location of production. It includes time-costs of commuting as well as the effective spending on transportation.%\footnote{It could also incorporate an income reduction if it is harder to find a job when living further away from the location of production.} 

Since spatial frictions increase with $\ell_k$, urban workers locate as close as possible to $\ell_k=0$. If one denotes $\ell_k=\phi_k$ the furthest away location of an urban worker, $\phi_k$ is endogenous in our framework and represents the fringe of city $k$.\footnote{Regions are assumed large enough in area such that cities do not expand in neighboring regions. $S$ is large enough such that for all cities, $\phi_k<\sqrt{S/\pi}$.} Workers residing in locations beyond $\phi_k$ produce the rural good and do not face spatial frictions, as rural workers do not commute.

We use the functional form $\tau(\ell_k)= a\cdot (w_{u,k})^{\xi_w}(\ell_k)^{\xi_\ell}$, $a>0$, $\xi_w \in (0,1)$ and $\xi_\ell \in (0,1)$, for which we provide in Appendix \ref{B-sec:extensions-commute} a micro-foundation through a commuting choice model. This modeling approach helps mapping commuting costs into observables from commuting data, but results do not depend qualitatively on the micro-foundation as long as commuting costs are increasing and concave in the opportunity cost of time and commuting distance. The concavity, $\xi_w \in (0,1)$ and $\xi_\ell \in (0,1)$, arises from the micro-foundation, whereby individuals optimally choose their commuting speed depending on their location $\ell_k$ and opportunity cost of time (wage rate $w_{u,k}$). This is important as it implies that, for a given residential location, the share of resources devoted to commuting falls with rising urban productivity and wages. In equilibrium, this makes individuals willing to live further away in order to enjoy larger homes. This is the channel through which rising urban productivity leads to faster commutes and suburbanization. The micro-foundation of commuting costs also enlightens the calibration as the elasticity of commuting costs to commuting distance (resp. income) is directly tied to the elasticity of commuting speed to commuting distance (resp. income), both of which have data counterparts. 

\textit{Remarks.} The spatial structure calls for a number of important remarks. First, if it were possible for all workers to locate at $\ell_k=0$, there would be no spatial frictions. Second, one should note that for $\ell_k \leq \phi_k$, land will be used for residential purposes to host urban workers. As a consequence, land available for rural production would also be maximized if all workers could locate at $\ell_k=0$. This case would correspond to an entirely `vertical' city, where land use and spatial frictions are irrelevant. We view this extreme case as a standard two-sector model of structural transformation.
%(added in extension \ref{B-sec:agglo}). %---the reason why a more compact city (lower $\phi$) always saves on commuting costs. %We allow for congestion and agglomeration effects in Appendix \ref{A-sec:agglomeration}.

\subsection{Household Optimization Conditions}
\label{subsec:HHopt}

We consider ex-ante identical workers simultaneously choosing their consumption expenditures and their location, taking all prices as given.

%\textbf{Commuting Choice Optimization.} Commuting costs in location $\ell_k$, $\tau(\ell_k)$, are the sum of spending on commuting using transport mode $m$, $f(m)$, and time-costs proportional to $w_{u,k} \cdot t(\ell_k)$, where $t(\ell_k)$ denotes the time spent on daily commutes of an individual located in $\ell_k$, such that
%\begin{equation}
%\tau(\ell_k)=f(m)+\zeta w_{u,k} \cdot t(\ell_k), \label{eq:tau1}
%\end{equation}
%where $0<\zeta \leq 1$ represents the valuation of commuting time in terms of foregone wages. Transportation modes $m$ are continuously ordered by their speed, as in \cite{desalvo1996income}, such that $m$ denotes both the mode and the speed of commute. The commuting time (both ways) is therefore, $t(\ell_k)=\frac{2\ell_k}{m}$. Faster commutes are more expensive and $f(m)$ is increasing in $m$. For tractability, we use the following functional form, $f(m)=\frac{c_\tau}{\eta_m} m^{\eta_m}$, with $\eta_m>0$ and $c_\tau$ a cost parameter measuring the efficiency of the commuting technology. This expression for commuting costs facilitates parametrization and preserves some tractability, while elucidating the main mechanisms.\footnote{The cost $f(m)$ has several possible interpretations. At a more macro level, it can represent the fixed cost of installing public transportation, where a faster mode is more expensive (a train line versus the horse drawn omnibus). At a more individual level, it represents the cost of buying an individual mean of transportation---a bike being cheaper than an automobile. However, this reduced-form approach sets aside the possibility that the implemented commuting technologies and their speed depend in a more sophisticated way on the equilibrium allocation in the city (e.g. traffic congestion or the construction of transport infrastructures may depend on the spatial allocation of urban residents).} %The quantitative Section \ref{sec:QM} uses a more general function for the spending on commuting $f$, also increasing in the commuting distance $\ell$ and urban wages $w_u$, $f=f(m,\ell,w_u)$.

%At any given moment in time, prevailing technology offers different transportation modes ordered by their respective speed $m$. An individual in location $\ell_k$ chooses the mode of transportation of speed $m$ in order to minimize the commuting costs $\tau(\ell_k)$. By equalizing the marginal cost of a higher speed $m$ to its marginal benefits in terms foregone wage, the optimal chosen mode/speed satisfies,
%\begin{equation}
%m=\left(  \frac{2\zeta w_{u,k}}{c_\tau}\right)^{1-\xi}\cdot \ell_k^{1-\xi}, \label{eq:speed}
%\end{equation}
%where $\xi \equiv \frac{\eta_m}{1+\eta_m} \in (0,1)$. Individuals living further away choose faster commuting modes. The speed of commuting also increases with the wage rate as a higher wage increases the opportunity cost of time. Using Equations \eqref{eq:tau1} and \eqref{eq:speed}, we get that equilibrium commuting costs satisfy,
%\begin{equation}
%\tau(\ell_k)= a\cdot (w_{u,k}\ell_k)^{\xi}, \label{eq:tau3}
%\end{equation}
%where $a \equiv \left( \frac{1+\eta_m}{\eta_m}\right)  c_\tau^{\frac{1}{1+\eta_m}}\left( 2\zeta\right) ^{\frac{\eta_m}{1+\eta_m}}>0$.  Commuting costs increase with the wage rate (the opportunity cost of time) and the commuting distance with constant elasticities.\footnote{Commuting costs also fall with a better commuting technology (lower $a$). $a$ is alike a relative price of commuting: if technology improves relatively faster in the commuting sector, the relative price $a$ (in terms of urban goods) falls. Our baseline simulations will hold $a$ fixed focusing on urban productivity as the main driver of faster commutes.}  Since individuals optimally choose the commuting speed, the elasticity $\xi$ of commuting costs to the wage rate is strictly smaller than unity. This is important as it implies that, for a given residential location, the share of resources devoted to commuting falls with rising urban productivity and wages. In equilibrium, this makes individuals willing to live further away in order to enjoy larger homes. This is the channel through which rising urban productivity leads to faster commutes and suburbanization. Our derivation of commuting costs also enlightens the calibration as the elasticity of commuting costs to commuting distance (resp. income) is directly tied to the elasticity of commuting speed to commuting distance (resp. income), which have data counterparts. (Equation \eqref{eq:speed}).

\textbf{Budget Constraint and Expenditures.} Consumers earn a wage income net of spatial frictions  $w(\ell_k)$ in location $\ell_k$ of region $k$. Given the spatial structure, $w(\ell_k)=w_{u,k} - \tau(\ell_k)$ for $\ell_k \leq \phi_k$ and $w(\ell_k)=w_{r,k}$ for $\ell > \phi_k$, where $w_{r,k}$ denotes the wage rate in the rural sector of region $k$. Consumers also earn land rents, $r$. Land rents are redistributed lump-sum equally and are thus assumed to be independent of location. Defining $p$ as the relative price of the rural good in terms of the numeraire urban good, the budget constraint of a worker in location $\ell_k$ of region $k$ satisfies%
\begin{equation}
pc_r(\ell_k)+c_u(\ell_k)+q(\ell_k)h(\ell_k)=w(\ell_k)+r,  \label{eq:BC}
\end{equation}%
with $q(\ell_k)$ the rental price per unit of housing (or housing price) in location $\ell_k$ of region $k$.

Maximizing utility (Equation \eqref{eq:U}) subject to the budget constraint (Equation \eqref{eq:BC}) yields the following consumption expenditures,
\begin{eqnarray}
pc_r(\ell_k)=(1-\gamma)\nu\left(\frac{p}{P}\right)^{1-\sigma}(w(\ell_k)+r + \underline{s} -p\underline{c})+p\underline{c} \label{eq:expenditurer}\\
c_u(\ell_k)=(1-\gamma)\left( 1-\nu\right)\left(\frac{1}{P}\right)^{1-\sigma}  (w(\ell_k)+r + \underline{s} -p\underline{c}) - \underline{s} \label{eq:expenditureu} \\
q(\ell_k)h(\ell_k)=\gamma (w(\ell_k)+r+ \underline{s}-p\underline{c}), \label{eq:expenditureh}
\end{eqnarray}
with the composite price index of urban and rural goods, $
P=\left[ \nu p^{1-\sigma}+\left( 1-\nu\right)\right]^{\frac{1}{1-\sigma}}$.
Due to the presence of subsistence needs ($\underline{c}>0$), individuals reallocate consumption away from the rural good with rising income, increasing the consumption share of the urban good and housing (income effects). The reallocation of demand towards the urban good is stronger when $\underline{s}>0$. The elasticity $\sigma$ parametrizes substitution effects between rural and urban consumption, vanishing for $\sigma=1$.

\textbf{Mobility Equations and Sorting.} Since the rural and the urban good are perfectly tradable, urban workers in city $k$, which would all prefer locations closer to $\ell_k=0$, compete for these locations. Adjustment of housing prices through the price of land makes sure that households remain indifferent across different locations in a given region $k$. Using Equations \eqref{eq:expenditurer}-\eqref{eq:expenditureh}, this implies the following mobility equation, where consumption is equalized to $\overline{C}_k$ across locations $\ell_k$,
\begin{equation}
\overline{C}_k=C(\ell_k)=\kappa\frac{w(\ell_k)+r+ \underline{s}-p\underline{c}}{q(\ell_k)^{\gamma}}, \label{Eq:log-indiff}
\end{equation}
with $\kappa$ constant across locations, equal to $\left( (1-\gamma)\nu\right) ^{\left( 1-\gamma\right) \nu}\left( (1-\gamma)(1-\nu)\right) ^{\left( 1-\gamma\right) \left( 1-\nu\right) }\gamma^{\gamma}/P^{1-\gamma}$.

Equation \eqref{Eq:log-indiff} implies that $\left( \frac{w(\ell_k)+r+ \underline{s}-p\underline{c}}{q(\ell_k)^{\gamma}}\right)$ is constant across locations in region $k$. This holds within urban locations ($\ell_k \leq \phi_k$), within (identical) rural locations, as well as when comparing an urban and rural worker. Since workers in the rural sector do not face spatial frictions and live in ex-post identical locations, $\ell_k \geq \phi_k$, the price of housing must be the same across these locations. We denote by $q_{r,k}$ the price of housing in the rural sector of region $k$, where $q_{r,k}=q(\ell_k \geq \phi_k)$. A worker in the rural sector earns a wage $w_{r,k}$, receives land rents $r$ and faces the same housing price $q_{r,k}=q(\phi_k)$ than an urban worker at the fringe. Therefore, we have
\begin{equation}
w(\phi_k)=w_{r,k}=w_{u,k} -\tau(\phi_k). \label{eq:mobUR}
\end{equation}
In other words, the urban worker at the urban fringe must have the same wage net of commuting frictions than a rural worker---commuting frictions generating an urban-rural wage gap. Equation \eqref{eq:mobUR} is essential to understand the spatial allocation of workers: higher spatial frictions at the fringe $\phi_k$ reduce incentives of rural households to move to the city. 


Within city locations ($\ell_k \leq \phi_k$), the housing price adjusts such that workers are indifferent across locations of city $k$. Using Equations \eqref{Eq:log-indiff} and \eqref{eq:mobUR}, we get a housing rental price gradient:
\begin{equation}
q(\ell_k)=q_{r,k} \left( \frac{w(\ell_k)+r+ \underline{s}-p\underline{c}}{w(\phi_k)+r+ \underline{s}-p\underline{c}}\right)
^{1/\gamma}=q_{r,k} \left( \frac{w(\ell_k)+r+ \underline{s}-p\underline{c}}{w_{r,k}+r+ \underline{s}-p\underline{c}}\right)
^{1/\gamma}, \label{eq:q_d}
\end{equation}
Within city $k$, $q(\ell_k)$ is falling with $\ell_k$ to compensate workers living in worse locations. For $\ell_k$ above $\phi_k$, the housing price is constant, equal to $q_{r,k}$. A crucial difference compared to the standard urban model is that the fringe price $q_{r,k}$ is endogenously determined in our general equilibrium model.

Workers can freely move across regions, therefore equalizing the composite consumption $\overline{C}_k$ of the urban and rural worker at the fringe across the different regions. For all regions $k \in \{1,...K\}$,
\begin{equation}
\overline{C}_k=\overline{C}=\kappa\frac{w_{u,k}-\tau(\phi_k) + r + \underline{s} -p\underline{c}}{(q_{r,k})^{\gamma}}=\kappa\frac{w_{r,k}+r+ \underline{s}-p\underline{c}}{(q_{r,k})^{\gamma}}.
\label{eq:log-indiff-acrossk}
\end{equation}
Equations \eqref{Eq:log-indiff} and  \eqref{eq:log-indiff-acrossk} guarantee that workers are indifferent between locations within and across regions.

\subsection{Producers' Optimization Conditions}
\label{subsec:PPopt}

Goods producers choose the amount of labor, and land for the rural producer, while land developers choose the supply of housing space in each location $\ell_k$, to maximize profits, taking all prices as given.

\textbf{Urban and Rural Factor Payments.} Perfect competition ensures that the urban wage in each region $k \in \{1,...K\}$ is,
\begin{equation}
w_{u,k}=\theta_{u,k}. \label{eq:mpu}
\end{equation}
%in terms of units of the urban good, which is used as numeraire.  

Rural workers and land are paid their marginal productivities in each region  $k \in \{1,...K\}$,
\begin{eqnarray}
w_{r,k}=\alpha p\theta_{r,k}\left(\frac{S_{r,k}}{L_{r,k}}\right)^{1-\alpha}, \label{eq:foc-w}\\
\rho_{r,k}=(1-\alpha)p\theta_{r,k} \left(\frac{L_{r,k}}{S_{r,k}}\right)^{\alpha}, \label{eq:foc-q}
\end{eqnarray}
where $\rho_{r,k}$ is the rental price of land anywhere in the rural sector of region $k$.

\textbf{Housing Supply.} Profits per unit of land of the developers are in each location $\ell_k$ of region $k$,
$$\pi(\ell_k)=q(\ell_k)H(\ell_k)-\frac{H(\ell_k)^{1+1/\epsilon}}{1+1/\epsilon}-\rho(\ell_k),$$
where $\rho(\ell_k)$ is the rental price of a unit of land in location $\ell_k$ (the land price). Maximizing profits gives the following supply of housing $H(\ell_k)$ in a given location $\ell_k$,
\begin{equation}
H(\ell_k)=q(\ell_k) ^{\epsilon}, \label{eq:Hsupply}
\end{equation}
where the parameter $\epsilon$ is the price elasticity of housing supply. More convex costs to build intensively on a given plot of land reduces the supply response of housing to prices. Our framework allows to consider location-specific housing supply elasticities $\epsilon(\ell_k)$ as a straightforward extension---housing supply response might be constrained in some locations (natural constraints, regulations, ...).

\textbf{Residential Land Prices.} Lastly, free entry implies zero profits of land developers. This pins down land prices in a given location,
\begin{equation}
\rho(\ell_k)=\frac{q(\ell_k)H(\ell_k)}{1+\epsilon}=\frac{q(\ell_k)^{1+\epsilon}}{ 1+\epsilon}. \label{eq:q_r_supply}
\end{equation}

Equation \eqref{eq:q_r_supply}, together with Equation \eqref{eq:q_d}, implies that land prices are also higher in locations closer to the city center, more so if land developers can build more intensively (higher $\epsilon$). And, for locations beyond the fringe $\phi_k$ of city $k$, the land price is constant, $\rho_{r,k} = \rho(\ell_k \geq \phi_k)$, as for the housing price $q_{r,k}$.

Arbitrage across land use implies that the land price in the urban sector, $\rho(\ell_k)$, must in equilibrium be above the marginal productivity of land for production of the rural good (Equation \eqref{eq:foc-q}), where the condition holds with equality in the rural part of the region, for $\ell_k \geq \phi_k$,
\begin{equation}
\rho_{r,k}=\frac{q_{r,k}^{1+\epsilon}}{ 1+\epsilon }=(1-\alpha)p\theta_{r,k} \left(\frac{L_{r,k}}{S_{r,k}}\right)^{\alpha}. \label{eq:q_r_supply2}
\end{equation}
Importantly, this equation shows that a fall in the relative price of rural goods and/or a reallocation of workers away from the rural sector lowers the price of urban land at the city fringe. 


\subsection{Market Clearing Conditions} \label{sec:Mktclearing}
%In equilibrium, total demand of rural and urban goods is equal to supply and, in each location $\ell$ housing demand is equal to housing supply.

\textbf{Housing Market Equilibrium.}  Using Equations \eqref{eq:expenditureh} and \eqref{eq:q_d}, the demand for housing space per worker $h(\ell_k)$ in each location of city $k$ is increasing with $\ell_k$ for $\ell_k \leq \phi_k$,
\begin{equation}
h(\ell_k)=\gamma \left( \frac{w(\ell_k)+r+ \underline{s}-p\underline{c}}{q(\ell_k)}\right) =\left(
\frac{\gamma }{q_{r,k}}\right) (w(\phi_k)+r+ \underline{s}-p\underline{c})^{1/\gamma}
(w(\ell_k)+r+ \underline{s}-p\underline{c})^{1-1/\gamma}. \label{eq:Hdemand}
\end{equation}
Facing higher housing prices, households closer to the CBD demand less housing space. A lower fringe price $q_{r,k}$ and lower spending for subsistence $p\underline{c}$ increase the demand for housing space in the city. In the rural area, housing demand per rural worker is constant, $h(\ell_k \geq \phi_k)=\gamma \left( \frac{w_{r,k}+r+ \underline{s}-p\underline{c}}{q_{r,k}}\right)$. 

Consider first locations within city $k$, $\ell_k \leq \phi_k$. Market clearing for housing in each location implies $H(\ell_k)=D_k(\ell_k)h(\ell_k)$, where $D_k(\ell_k)$ denotes the density (number of urban workers) in location $\ell_k$ of city $k$. The density $D_k(\ell_k)$ follows from Equations \eqref{eq:Hsupply} and \eqref{eq:Hdemand}, hence
\begin{equation}
D_k(\ell_k)=\frac{H(\ell_k)}{h(\ell_k)}=\frac{ q(\ell_k)^{1+\epsilon}}{\gamma(w(\ell_k)+r+\underline{s}-p\underline{c})}\label{eq:density}.
\end{equation}
Density for $\ell_k \leq \phi_k$ can be rewritten using Equation \eqref{eq:q_d} and Equation \eqref{eq:q_r_supply} as
\begin{equation}
D_k(\ell_k)=  \rho_{r,k} \frac{1+\epsilon}{\gamma}(w(\phi_k)+r+\underline{s}-p\underline{c}) ^{-\frac{1+\epsilon}{\gamma}} (w(\ell_k)+r+\underline{s}-p\underline{c})^{\frac{1+\epsilon}{\gamma} -1}.\label{eq:density2}
\end{equation}
Importantly, a lower rural land price $\rho_{r,k}$ at the urban fringe of city $k$ lowers density across all urban locations of the city. Integrating density defined in Equation \eqref{eq:density2} across urban locations of city $k$ gives the total urban population of the city,
\begin{equation}
L_{u,k}=\int_0^{\phi_k}D_k(\ell_k)2\pi \ell_k d\ell_k= \rho_{r,k} \int_0^{\phi_k}\frac{1+\epsilon}{\gamma}(w(\phi_k)+r+ \underline{s}-p\underline{c}) ^{-\frac{1+\epsilon}{\gamma} } (w(\ell_k)+r+ \underline{s}-p\underline{c})^{\frac{1+\epsilon}{\gamma} -1}2\pi \ell_k d\ell_k. \label{eq:citysize}
\end{equation}
Equation \eqref{eq:citysize} pins down the city size $\phi_k$. It says that if more workers are willing to move to the urban sector, city will have to be bigger in area to host them---$\phi_k$ is increasing with $L_{u,k}$. %One should notice that the city's area increases with a lower the price of land $\rho_r$ at the fringe, lower subsistence needs $p\underline{c}$ and lower commuting frictions $\tau(\ell)$.

In the rural area of region $k$, in locations $\ell_k \geq \phi_k$,
\begin{equation*}
L_{r,k} \gamma \left( w_{r,k}+r+\underline{s}-p\underline{c}\right)=S_{hr,k} \left(q_{r,k}\right)^{1+\epsilon}=S_{hr,k}(1+\epsilon)\rho_{r,k},
\end{equation*}
where $S_{hr,k}$ is the amount of land demanded in the rural area for residential purposes in region $k$.
%This leads to the following demand of land for residential purposes in the rural area,
%\begin{equation}
%S_{hr}=\frac{L_r \gamma \left( w_r+r+ \underline{s}-p\underline{c}\right)}{(1+\epsilon) \rho_r}. \label{eq:Shr}
%\end{equation}

\textbf{Land and labor market clearing.} Land is used for residential or productive purposes. With total land available in fixed supply $S$ in each region $k$, the land market clears locally in all regions $k \in \{1,..., K\}$,
\begin{equation}
S_{r,k}+S_{hr,k}+\pi\phi_k^2 = S \label{eq:landmarketclearing}
\end{equation}
with the demand of land for housing in the rural area of each region $S_{hr,k}$ equal to $\frac{L_{r,k} \gamma \left( w_{r,k}+r+ \underline{s}-p\underline{c}\right)}{(1+\epsilon) \rho_{r,k}}$.

The labour market clears globally. The labor market clearing is such that the total population $L$ is located either in the city or in the rural area of a region $k$,
\begin{equation}
\sum_{k=1}^{K}L_k=\sum_{k=1}^{K}(L_{r,k}+L_{u,k})=L. \label{eq:labormarketclearing}
\end{equation}
%\textbf{Land rents.} 
Aggregate land rents, $rL$, include the land rents generated both in the city and in the rural area of each region $k$,
%\begin{equation*}
%rL=\sum_{k=1}^K\left(\int_{0}^{\phi_k} \rho_k(\ell)2\pi\ell d\ell+\rho_{r,k} \times (S_{r,k} + S_{hr,k})\right).
%\end{equation*}
%This is equivalent to, using Eq. \ref{eq:landmarketclearing},
\begin{equation}
rL=\sum_{k=1}^K\left(\int_{0}^{\phi_k} \rho(\ell_k)2\pi\ell_k d\ell_k+\rho_{r,k} \times (S-\pi\phi_k^2)\right), \label{eq:rL}
\end{equation} 
where it is useful to notice that the rental income in the city exceeds the rental income of farmland for the same area due to spatial frictions.

% no longer valid with non-linear commuting costs:

% Following a well-known result from \cite{arnott1981aggregate}, we have that urban rents, on top of farmland rents, are equal to the aggregate losses due to spatial frictions $\tau_uL$, such that,
% \begin{equation}
% rL=\rho+\tau_uL_u, \label{eq:landrents}
% \end{equation}
% where $w_u\int_{0}^{\phi}\tau(l)D(l)d\ell=\tau_uL_u=\int_{0}^{\phi}(q(l)-\rho_r)d\ell$, where $\tau_u$ denotes the average loss per urban worker due to spatial frictions.



%to be used later
%Note that at the optimum, use of urban good as inputs to expand the housing space is equal to  $\frac{\epsilon(l)q(l)H(l)}{1+\epsilon(l)}$.

\textbf{Good markets clearing.} A last step consists in clearing the goods market for rural and urban goods. Rural and urban goods markets clear globally. The rural good is only used for consumption. The market clearing condition for rural goods is
%to pin down the allocation of labor across sectors for a given equilibrium city size $\phi$.
\begin{equation}
\sum_{k=1}^K C_{r,k}=\sum_{k=1}^K  Y_{r,k}\label{eq:mcrural},
\end{equation}
where $C_{r,k}=\left( \int_0^{\phi_k} c_{r,k}(\ell_k)D_k(\ell_k) 2\pi \ell_k d\ell_k+ c_{r,k}(\ell_k \geq \phi_k)L_{r,k}\right)$ denotes the total consumption of rural goods by urban workers (the first term) and rural workers (the second term) of region $k$.

The urban good market clearing condition is more involved as urban goods are either consumed, used as intermediary inputs to build residential housing (in all locations) or used to pay for commuting costs. As the condition will be verified by Walras law, the expression is relegated to Appendix \ref{B-sec:marketclearing} (Equation \eqref{B-eq:goods-market-clearing-k}).

%By summing demand for urban goods across all locations, the market clearing condition for urban goods is
%\begin{equation}
%\sum_{k=1}^K \left( C_{u,k} + \mathbb{T}_k + \mathbb{H}_{u,k} \right)= \sum_{k=1}^K  Y_{u,k} \label{eq:mcurban},
%\end{equation}
%where the terms of the summation in brackets denote, in order: 
%\begin{enumerate}
%	\item$C_{u,k}=\left( \int_0^{\phi_k} c_{u,k}(\ell)D_k(\ell) 2\pi \ell d\ell+ c_{u,k}(\ell_k \geq \phi_k)L_{r,k}\right) $ denoting total consumption of urban goods by urban workers (its first term) and rural workers (second term of $C_{u,k}$) of region $k$;
%	\item $\mathbb{T}_k = \int_0^{\phi_k} \tau_k(\ell)D_k(\ell)2\pi \ell d\ell$ denoting urban good used to pay for commuting costs. Notice that the amount of urban good used for commuting purpose or to produce housing is region-specific.
%	\item $\mathbb{H}_{u,k}=\left(\int_0^{\phi_k} \frac{\epsilon_k(\ell)}{1+ \epsilon_k(\ell)}q_k(\ell)H_k(\ell)2\pi \ell d\ell +\frac{\epsilon_r}{1+ \epsilon_r}q_{r,k}H_{r,k}\right)$ denotes the total demand of urban goods for urban housing (the first term) and rural housing (the second term) in region $k$.
%\end{enumerate}

 

%Let us introduce $y$ as the aggregate per capita income in the economy net of spatial frictions,
%\begin{equation*}
%y=r+\frac{L_r}{L}w_r+\frac{1}{L}\int_{0}^{\phi}w(\ell)D(\ell)d\ell.
%\end{equation*}
% Or equivalently, using the rural-mobility (equation \eqref{eq:mobUR}) and the equilibrium land rents Equation \eqref{eq:landrents},
% \begin{equation*}
% y=\frac{\rho_r}{L}+\frac{1}{L}L_{u}(w_{u}-\tau_{u}).
% \end{equation*}
% This last equation clearly shows how spatial frictions reduce $y$. In a frictionless economy, $\tau (\phi)=0$, aggregate income per capita is simply the sum of the rents from farmland and wages, where wages are equalized across space in equilibrium due to perfect mobility.
%With spatial frictions, aggregate income per capita is as if the wage of rural workers (a fraction $\frac{L_r}{L}$) was reduced by a fraction $\tau(\phi)$ relative to the frictionless economy.


%Aggregating Equations \eqref{eq:expenditurer}-\eqref{eq:expenditureu} across locations, we get that aggregate per capita consumption of rural good and urban good satisfy
%\begin{eqnarray*}
%	pc_r&=&\nu (1-\gamma)(y+ \underline{s}-p\underline{c})+p\underline{c}\\
%	c_u&=& (1-\nu)(1-\gamma)(y+ \underline{s}-p\underline{c}) - \underline{s}
%\end{eqnarray*}
%The rural good is only used for consumption. The rural good market clearing condition is,
%\begin{equation}
%\nu (1-\gamma)y+ \nu (1-\gamma)(\underline{s} - p\underline{c}) + p\underline{c}=py_r, \label{eq:mcrural}
%\end{equation}
%where $y_r=\frac{Y_r}{L}$ denotes the production per worker of the rural good.

%The urban good market clearing is more involved as urban goods are either consumed, used as intermediary inputs to build residential housing (in all locations) or used to pay for commuting costs. The sum of these three uses equals the supply of the urban good, expressed per capita,
%\begin{equation}
%c_u+\frac{1}{L}\int_{0}^{\phi}\tau(\ell)D(\ell)d\ell+\frac{1}{L}\frac{\epsilon}{1+\epsilon}\int_{0}^{S} q(\ell)H(\ell)d\ell=y_u,\label{eq:mcurban}
%\end{equation}
%where $y_u=\frac{Y_u}{L}$ denotes the production per worker of the urban good.
% Under the assumption of constant housing supply elasticity, i.e. $\epsilon(l) = \epsilon$, using equation \eqref{eq:q_r_supply} and the expression for per capita consumption on the urban good, the urban good market clearing further simplifies into
% \begin{equation}
% (1-\nu)(1-\gamma)(y+\underline{s}-p\underline{c})+\epsilon r=y_u(1-\tau_u). \label{eq:mcurban}
% \end{equation}
% The left-hand side represents the urban goods consumed (first term) or used as inputs for residential housing (second-term). Their sum is equal to the supply of urban goods net of spatial frictions.

\subsection{Equilibrium Definition} \label{subsec:Eqdef}

For a given set of exogenous parameters, technological parameters $(\theta_{u,k},\theta_{r,k}, \alpha)$, commuting cost parameters $(a, \xi_w, \xi_\ell)$ and resulting spatial frictions $\tau(\ell_k)$ at each location $\ell_k \in \mathcal{L}$, housing supply conditions $\epsilon$,  and preference parameters, $(\nu,\gamma, \underline{c},\underline{s}, \sigma)$, the equilibrium is defined as follows:
\begin{definition}
	In an economy with $K$ regions, an equilibrium is, in each region $k \in \{1, ..., K\}$, a sectoral labor allocation, $(L_{u,k},L_{r,k})$, a city fringe $\phi_k$ and rural land used for production $S_{r,k}$, sectoral wages $(w_{u,k},w_{r,k})$, a rental price of farmland $(\rho_{r,k})$ together with a relative price of rural goods $(p)$ and land rents $(r)$, such that:
	\begin{itemize}
		\item Workers are indifferent in their location decisions, within and across regions, Equations \eqref{Eq:log-indiff} and \eqref{eq:log-indiff-acrossk}.
		\item Factors are paid the marginal productivity in each region $k \in \{1, ..., K\}$, Equations \eqref{eq:mpu}-\eqref{eq:foc-q}.
		\item The demand for urban residential land (or the city fringe $\phi_k$) satisfies Equation \eqref{eq:citysize} in each region $k \in \{1, ..., K\}$.
		\item The land market clears in each region $k \in \{1, ..., K\}$, Equation \eqref{eq:landmarketclearing}. 
		\item The labor market clears globally, Equation \eqref{eq:labormarketclearing}.
		\item Land rents satisfy Equation \eqref{eq:rL}.
		\item The rural goods market clears globally, Equation \eqref{eq:mcrural}.
	\end{itemize}
\end{definition}

The main intuition for the equilibrium allocation goes as follows: in each city $k$, if the urban sector hosts more workers, the area of the city has to be larger ($\phi_k$ tends to increase with $L_{u,k}$). However, if the city is larger in area, the worker in the further away urban location commutes more, making the urban sector less attractive for workers: a higher $\phi_k$ reduces the incentives of workers to move from the rural to the urban sector of city $k$ ($L_{u,k}$ tends to decrease with an increasing $\phi_k$). Given technology, the combination of these two forces pins down the allocation of workers across sectors in each region, together with the land used for urban residential housing. Across regions, the allocation of workers is largely driven by differences in regional productivities---more productive regions hosting more workers. Since the equilibrium cannot be described analytically, we provide a simple numerical illustration in Appendix \ref{B-sec:numillus_new} to elucidate the main mechanisms through which increasing productivity in both sectors change the population, area and density of cities. This experiment sheds light on data moments that can be used to identify the model's parameters in the quantitative evaluation of Section \ref{sec:QM} and allows us to discuss the modelling assumptions which are important for the main model's implications. 



\subsection{Discussion} \label{sec:discussion}

\textbf{Preferences.} With sectoral productivity evolutions, structural change is driven by income effects due to non-homotheticities or by substitution effects for $\sigma \ne 1$. Focusing on income effects ($\sigma = 1$), rural productivity growth combined with subsistence needs for rural goods frees up land and labor for the urban sector to expand (`rural labor push')---the dominant driver of structural change for a large $\underline{c}$ relative to $\underline{s}$. As illustrated by the experiment in Appendix \ref{B-sec:numillus_new}, this perspective replicates qualitatively the salient facts described in Section \ref{sec:empevidence} for France regarding the expansion of the urban area, the evolution of urban density and land values. An alternative view would emphasize a rising demand for (luxury) urban goods as income rises (`urban labor pull')---corresponding to a high $\underline{s}$ relative to $\underline{c}$. While such a calibration can generate employment shares broadly in line with the data, it cannot replicate the observed reallocation of land use and the corresponding fall in urban density. For $\underline{s}>\underline{c}$, as income increases, the spending share on housing falls due to a low income elasticity of housing demand: workers are willing to reduce their housing size to consume more of the urban good. The city does not expand much in area to host more numerous urban workers and urban density might not fall. Importantly, the increase of the housing spending share in the data is informative regarding the relative magnitude of $\underline{c}$ and $\underline{s}$---a crucial insight for the joint calibration of these parameters. We investigate the role of substitution effects for $\sigma \ne 1$ in the quantitative evaluation of Section \ref{sec:QM}. In the context of France, the main insights are delivered when structural change is driven by income effects since agricultural and urban productivity largely increased at a similar rate in France since 1840 (see Figure \ref{fig:theta} below). 

\textbf{Rural Technology.} An important insight of the theory is the potential role of \textit{rural} productivity growth for urbanization and the reallocation of workers away from the rural sector (`rural labor push') but also to replicate the large decline in urban density, the fall in farmland prices (relative to income) and the reallocation of land rents towards urban areas. The difference in land intensity between sectors and the substitutability between land and labor in rural production are important for these implications. Intuitively, with a rural land intensity closer to the urban one, the farmland price would decrease less (relative to income) with structural change. As the opportunity cost of expanding the city is higher, this limits the rise in urban areas and the decline in urban density. Similarly, with an elasticity of substitution between land and labor in the rural sector above (resp. below) unity, the farmland price would decrease less (resp. more) with the reallocation of labor to the urban sector as investigated in Section \ref{sec:extensions}.\footnote{The rural production technology remains simple to focus on the core mechanisms. A more sophisticated production (with capital and/or factor biased technical change) could weaken or reinforce the results depending on the substitutability between factors and on the impact of technical change on land per worker. However, it is worth noting that, with commuting frictions, efficiency requires to reallocate labor more than land away from agriculture with structural change---leading to a rise in $S_r/L_r$ and a drop in $\rho_r$ (relative to income). Hence, our theory provides a complementary mechanism to technological explanations of the increase in land per worker in agriculture.}

\textbf{Urban Technology and Commuting Costs.} Urban production does not use land and is concentrated in the center. Relaxing only the first assumption is unlikely to change the main outcomes for a land intensity significantly smaller in the urban sector. However, with urban production using land, some activities could be reallocated in the suburbs since central land becomes more expensive as the city grows. With further away residents commuting less, urban density could decline even more. While endogenizing firms and workers location remains a difficult task, we partly capture these mechanisms in a later extension where we relax the monocentric assumption---assuming that commuting distance does not map one for one with residential distance (Section \ref{sec:extensions}). In this latter Section, we also consider congestion and agglomeration forces absent from the baseline theory. Finally, an important assumption implied by the micro-founded commuting choice model is the concavity of commuting costs with respect to distance and urban wage, $\xi_\ell<1$ and $\xi_w<1$. While not necessary, these assumptions appear sufficient to guarantee a drop in urban density in numerical experiments, but less concave commuting costs (higher $\xi_\ell$ or $\xi_w$) would limit the increase in urban area and the fall in density.\footnote{Our approach implicitly assumes that commuting time is taken out of working time entirely. Results would be similar in a framework where commuting time also partly reduces leisure time if leisure is valued at the wage rate.} In particular, the magnitude of the income elasticity of commuting costs, $\xi_w$, matters quantitatively for urban sprawl driven by \textit{urban} productivity growth: facing higher urban wages, urban residents have stronger incentives to relocate in the suburbs to enjoy larger homes when commuting costs increase less with income (a lower $\xi_w$). 

\textbf{Land use and housing regulations.} Our theory abstracts from land use and housing regulations, which would distort equilibrium prices and the equilibrium allocation. Stricter land use regulations aimed at preserving the rural area would limit the expansion of urban areas. This would imply higher urban housing prices together with a higher urban density. While such regulations are currently in place in France, they became effective only in the most recent decades. To the contrary, stricter housing regulations limiting the housing supply in some locations would make cities expand more in area and, consequently, decrease urban density. Such regulations are investigated in a reduced-form way in the quantitative evaluation of Section \ref{sec:QM}, where the housing supply elasticities are assumed to be lower in the central parts of cities than in the suburbs or the rural part of the economy. This is meant to capture that it is cheaper to build closer to the city fringe than in the city center.



\end{document}