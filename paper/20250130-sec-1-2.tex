\documentclass[./20250130-paper.tex]{subfiles}


\begin{document}
	
	Since the early years of the industrial revolution, population massively migrated from rural areas towards cities. This widespread phenomenon of urbanization went together with the reallocation of workers away from the agricultural sector towards manufacturing and service sectors---a phenomenon of structural change. How do cities grow when these well-known phenomena occur? Cities can become denser for a given area---growth at the intensive margin. They can also become larger in surface to accommodate more workers---via growth at the extensive margin.  Over a long period, cities have been growing essentially in area, at such a fast speed that their average density has been falling. In other words, over time, cities expanded faster in area than in population. We precisely document this stylized fact for France since 1870 but it is also documented on a global scale in \cite{angel2010persistent} for the recent period. In France, the population of the main cities has been multiplied by almost 4 since 1870, while their area increased by a factor of 30: the average urban density has thus been divided by a substantial factor of about 8. This paper shows that this persistent decline in density, despite the process of urbanization, is well explained by conventional theories of structural change with non-homothetic preferences and augmented with endogenous land use---whereby land can be used for agriculture or urban housing.
	
	A crucial insight of our theory is to consider that the value of agricultural land at the urban fringe determines the opportunity cost of expanding the area of cities for housing purposes. With low agricultural productivity, agricultural goods and farmland are expensive. High agricultural land values make cities initially small in area and very dense as households cannot afford large homes---a manifestation of the `food problem' (\cite{schultz1953}). With structural change driven by rising productivity, workers move away from rural areas towards cities, freeing up agricultural land. As the land value at the urban fringe falls relative to income and richer households start being able to buy larger homes, cities expand in area at a fast rate. Together with the reallocation of workers across sectors, reallocation of land use occurs---from agricultural use to urban use. We document that for France, since 1840, about 15\% of French land has been converted away from agricultural use. Our theory can account not only for the reallocation of factors away from agriculture but also for the faster growth of cities in area relative to population and the corresponding decline in average urban density---providing a novel mechanism explaining urban sprawl and suburbanization. This complements the traditional Urban Economics view that cities have sprawled following improvements in commuting technologies, which have allowed households to live further away from their workplace (see references in \cite{glaeser2004sprawl}, \cite{heblichreddingsturm2018}, \cite{redding2021}). 
	
	Our framework also provides novel predictions regarding the historical evolution of land values, which are in line with the evidence in \cite{piketty2014capital}. The value of farmland as a share of income, initially high due to subsistence needs, falls over time with structural change, while the value of urban land rises significantly. Moreover, despite rising housing demand, the fast expansion of cities at the extensive margin due to structural change initially limits the increase in urban land rents and housing prices. When the reallocation of workers and land out of agriculture slows down, the value of land must adjust to prevent further expansion of cities with rising workers' incomes and housing demand. Land values start to increase at a faster rate. Our theory thus predicts relatively flat land and housing values for decades before shooting up---a prediction which resembles very much the data for France and most advanced economies as best illustrated in \cite{knoll2017no}. Therefore, our theory provides novel insights on the joint evolution of the density of cities and land values along the process of economic development. It also helps understanding how the structure of cities, e.g. their urban extent and density, evolves with the process of structural transformation, shedding new light on the origins of urban sprawl.% in the process of economic development.%---a central matter in the artificialization of soils and their environmental impact (\cite{roy2018sustainable}).
	
	The contribution of our paper is threefold. First, we document new stylized facts on land use and urban expansion for France since the mid-nineteenth century. In particular, using historical maps and satellite data for the more recent period, we document the historical decline of the density of French cities. Between 1870 and 1950, the average density was divided by about 3 and again by about 2.5 until 1975---the thirty years post-World War II being characterized in France by faster structural change  and \textit{rural exodus} (\cite{mendras1970}, \cite{bairoch1989}, \cite{toutain1993production}). Together with the slowdown of structural change in the more recent decades, average urban density did not fall much since. Using novel cross-sectional data on local farmland values, we also show that, in recent times, cities surrounded by more expensive farmland are denser---confirming that the opportunity cost of building at the urban fringe matters for urban sprawl. These novel facts, together with the historical evolution of urban and agricultural land values in France, motivate our theory. 
	
	The second contribution is to develop a spatial general equilibrium model of structural change with endogenous land use and multiple cities/regions differing in their productivities. The production side features three sectors: rural, urban and housing. The rural (urban) sector produces agricultural (non-agricultural) tradable goods, the production of the agricultural good being more land intensive. The housing sector produces location-specific housing units using the urban good and land in the process. Land is in fixed supply and land use is rivalrous: land is either used for agriculture or for housing. Following the traditional monocentric model (\cite{alonso1964location}, \cite{muth}, \cite{mills1967aggregative}), urban land use (cities) emerges endogenously around given city centers due to commuting costs for workers: urban land is more densely populated than rural land and the urban fringe corresponds to the longest commute of a worker producing urban goods. Due to commuting frictions, urban workers are also compensated with a higher wage than rural workers. Importantly, the rental price of land at the fringe of each city must be equalized across potential usages---the marginal productivity of land in the rural sector determining the opportunity cost of expanding further urban land. The last important components of our theory are the drivers of structural change. Structural change is driven by the combination of non-homothetic CES preferences on the demand side, particularly a subsistence consumption for the rural good, and increasing productivity on the supply side. These ingredients generate transitory dynamics with rising productivity in agriculture at the heart of our story: in the old times, due to low agricultural productivity, land is scarce with high values of farmland with respect to income. Moreover, households devote a large fraction of their resources to feed themselves and cannot afford large homes. Few urban workers are concentrated on a small area and urban land is highly densely populated. Later on, with agricultural development, farmland is getting less valuable, accommodating rising demand for housing of more numerous urban workers. The city sprawls and average urban density might fall through two channels: the fall in the rental price of farmland (relative to income) at the urban fringe and the increasing share of spending towards housing. Note that this decline in urban density can occur even without improvements in commuting technology. Building upon \cite{leroy1983paradise} and \cite{desalvo1996income}, we account for the latter, more standard, mechanism by parametrizing a model of commuting mode choice, where individuals optimally choose faster commuting modes to live further away from the city center when urban wages increase. Thus, although the mechanisms are entirely different, both urban \textit{and} rural productivity growth can lead to sprawling and suburbanization. %However, the implications for density across urban locations are different. Increasing urban productivity and faster commutes lead to a reallocation of urban workers away from the center towards the city fringe. As a consequence, central density falls more than average urban density since suburban density increases. To the contrary, increasing agricultural productivity and structural change lead to the addition of lower and lower density settlements at the fringe of cities: suburban density falls more than the average urban density. While central density did fall since the mid-nineteenth century, historical data for Paris shows that it fell less than the average urban density. This suggests that both channels---the structural change and the commuting speed channels---have been playing a role in driving the density decline. 
	
	%We also show that agricultural productivity growth and structural change are crucial to understand the fall in agricultural land rents to the profit of urban ones. If land reallocation away from agriculture towards urban use was driven by urban productivity growth and faster commutes, rural land would be getting scarcer and more valuable: the value of farmland rents would increase, as a share of income. Agricultural land rents would also become relatively more important than urban ones---predictions that are widely counterfactual to the evidence in \cite{piketty2014capital}. %Quite differently, structural change driven by increasing rural productivity frees up farmland, lowering its value relative to income and reducing the importance of agricultural land rents to the profit of urban ones. These predictions are much more in line with the data. 
	
	%At the latest stages of the transition, in more recent times, the reallocation of workers and land use slows down. Urban expansion slows, urban density declines less and land prices increase more with rising productivity. Land becomes also particularly scarce in some preferential locations of very spread out cities.
	
	
	%We also emphasize how the supply side characteristics of the housing sector are essential quantitatively for our mechanisms: with persistently high housing supply elasticities across locations, urban land sprawls less despite more workers leaving the rural sector,  urban density declines significantly less and housing prices increase much less in the later stages of the transition.%
	
	
	The third contribution is to evaluate the quantitative ability of the spatial equilibrium model to replicate the reallocation of land use and land values in France since 1840. Using data from various historical sources, we measure sectoral factors of production and productivities since 1840 and calibrate the model to fit the process of structural change in France. Historical spatial data on farmland values and urban population discipline the spatial distribution of urban and rural productivity across regions/cities. To account for the use of faster commutes over time, we make use of a tractable parametrization of commuting costs and calibrate the elasticities of commuting speed to urban income and commuting distance using individual commuting data. We show that the model's predictions match relatively well the joint evolution of the urban extent, population density and land value over time and space. More specifically, our framework accounts for most of the decline in average urban density as well as the land value reallocation from rural to urban, and about half of the rise in housing prices since the mid-nineteenth century. Using cross-sectional data on local farmland prices and accounting for possible endogeneity issues, we find that higher farmland values at the urban fringe makes cities relatively denser---a prediction at the heart of our mechanisms. Quantitatively, the elasticity of urban density with respect to the farmland price found in cross-sectional data is in line with its model counterpart. Finally, we disentangle the importance of falling commuting costs relative to our novel mechanism based on structural change in explaining the density of urban settlements. First, we show that without structural change, one cannot match the decline in urban density---emphasizing the key role of improvements in agricultural productivity for urban sprawl. Second, when combined with structural change, the effect of faster commutes is magnified and remains quantitatively crucial to account for the density decline---without faster commutes, the model-predicted density decline over the period would be about 30\% of our baseline and short of the data. Third, faster commutes lead to a reallocation of urban workers from the center to the suburbs: central density falls more than average urban density since suburban density increases. To the contrary, structural change leads to the addition of lower and lower density settlements at the urban fringe: suburban density falls more than the average one. While central density did fall since 1870, historical data for Paris shows that it fell less than the average. Our quantitative predictions line up with the Parisian evidence suggesting that both channels---the structural change and the commuting speed channels---are necessary to account for the observed density decline. 
	% and cannot account on their own for the decline in density at the suburban fringe.
	
\textbf{Related literature.} The paper relates to several strands of literature in macroeconomics and spatial economics. From a macro perspective, it relates to the literature linking productivity changes and land values, starting with \cite{ricardo1817}. This traditional view would imply that a fixed factor such as land should rise in value with economic development (see, among others, \cite{nichols1970land} and \cite{grossman2017house})---a counterfactual prediction given historical measurement of housing prices and land values (\cite{piketty2014capital}, \cite{knoll2017no}, \cite{davisheathcote} for related U.S. evidence). An alternative view argues that the rise in land prices can be mitigated by improvements in commuting technologies (\cite{milessefton2020}). Our approach, in the tradition of the theory of structural change (\cite{herrendorf2014growth}), argues that farmland used to be valuable when agricultural productivity was low, but technological improvements can alleviate pressure on land. In a sense, our theory reconciles these different views in a unified spatial framework---adding endogenous land use and a housing sector to the most conventional multi-sector model with non-homothetic preferences (\cite{kongsamut}, \cite{gollin2007food}, \cite{herrendorf2013aer}, \cite{boppart2014structural}, \cite{comin2015structural}, \cite{alder2019theory}). While structural change and urbanization are known to be tightly linked (\cite{lewis1954economic}), the spatial dimensions have been rarely investigated. \cite{michaelsrauchredding}, \cite{eckertpeters2018} and  \cite{pijoan-mas-budi} are notable exceptions. The crucial difference to those is the ability of our framework to replicate the evolution of population density within locations, putting emphasis on the internal structure and density of cities, while their focus is on the distribution of population and the sectoral specialization across regions. We also emphasize the implications for land values and show how our framework can generate a sizeable urban-rural wage gap due to commuting frictions---a complementary explanation to the `agricultural productivity gap' (\cite{gollin2014agricultural}), different from migration costs or selection of migrants (\cite{restuccia2008agriculture}, \cite{lagakos2013waugh}, \cite{young2013inequality}).    
	
%	\textbf{Related literature.} The paper relates to several strands of literature in macroeconomics and spatial economics. From a macro perspective, it relates to the literature linking productivity changes and land values, starting with \cite{ricardo1817}. This traditional view would imply that a fixed factor such as land should continuously rise in value with economic development (see, among others, \cite{nichols1970land} and \cite{grossman2017house} for a recent contribution). However, such a prediction would not fit well the measurement of housing prices and land values over a long period as in \cite{piketty2014capital} and \cite{knoll2017no} (see also \cite{davisheathcote} for related U.S. evidence). An alternative view developed in \cite{milessefton2020} argues that the rise in land and housing prices can be mitigated by improvements in commuting technologies, which allow cities to expand outwards. Our approach, in the tradition of the theory of structural change, also argues that land used to be scarce and valuable while agricultural productivity was low, but that improvements in technology alleviate pressure on land, decreasing its value. In a sense, our theory reconciles these different views in a unified framework. From a theoretical perspective, we contribute to the literature on structural change, surveyed in \cite{herrendorf2014growth}, by considering a spatial dimension---adding endogenous land use and a housing sector---in the most conventional multi-sector model with non-homothetic preferences (\cite{kongsamut}, \cite{gollin2007food}, \cite{herrendorf2013aer}, \cite{boppart2014structural}, \cite{comin2015structural}, \cite{alder2019theory}). Structural change and urbanization are known to be tightly linked (\cite{lewis1954economic}). However, the literature has rarely investigated the spatial dimension of structural change, largely abstracting from spatial frictions. \cite{michaelsrauchredding}, \cite{eckertpeters2018} and  \cite{pijoan-mas-budi} are notable exceptions. The crucial difference to those is the ability of our framework to replicate the evolution of population density within locations, putting emphasis on the internal structure and density of cities, while their focus is more on the distribution of population and the sectoral specialization across regions. We also emphasize the implications for land values across time and space, largely absent in these studies. We also show how commuting frictions and location-specific land values generate a sizeable wedge between the workers marginal productivities in the rural and urban sector, an `agricultural productivity gap' (\cite{gollin2014agricultural})---a complementary explanation to urban-rural wage gaps, different from migration costs or selection of migrants towards cities (\cite{restuccia2008agriculture}, \cite{lagakos2013waugh}, \cite{young2013inequality}).
	
	Our paper also contributes to the literature in spatial economics on urban expansion surveyed in \cite{duranton2014growth,duranton2015urban}, where commuting costs shape urban density. We add an endogenous sectoral allocation of factors and a general equilibrium structure at the heart of the macro literature. Importantly, the land price at the urban fringe becomes an endogenous object itself affected by structural change. Related work in \cite{brueckner1990analyzing}, surveyed in \cite{bruecknerLall}, shows how location-specific land values pin down rural-urban migrations. However, without structural change and endogenous farmland prices, this approach stays quite silent regarding the dynamics of urbanization and land values. In the latter dimension, we contribute to explanations of land values across space (\cite{glaeser2005manhattan}, \cite{albouy2016cities}, \cite{albouyehrlichshin2018}, \cite{combes2018costs}). In the French context, we also relate to the historical measurement of land use in \cite{combesetal2021}.  Lastly, our paper contributes to quantitative spatial economics (\cite{ahlfeldt2015economics}, \cite{redding2017quantitative}) by emphasizing the extensive margin of cities.
	
	%add: heblichreddingsturm2018 https://www.nber.org/papers/w25047.bib
	
	The paper is organized as follows. Section \ref{sec:empevidence} provides motivating empirical evidence on land use, land values, urban expansion and population density across space over a long period in France. Section \ref{sec:baselinemodel} provides a spatial general equilibrium model of land use and structural change. Section \ref{sec:QM} evaluates quantitatively the model calibrated to French historical data. Section \ref{sec:conclusion} concludes.
	

\section{Historical Evidence from France}\label{sec:empevidence}



\subsection{Land use and Employment in Agriculture}
\textbf{Data.} Using various sources described in Appendix \ref{A-sec:aggregate}, we assemble aggregate data on employment shares in agriculture and agricultural land use in France since 1840. Historical data on land use in agriculture are available roughly every 30 years (or less) until the 1980s and then at higher frequency. They are largely extracted from secondary sources based on the Agricultural Census (Recensement Agricole), and cross-checked with various alternative historical sources (\cite{toutain1993production} among others). Post-1950, data are from the Ministry of Agriculture.


\textbf{Employment.} As all countries going through structural transformation, France exhibits significant reallocation of labor away from agriculture over the period, from about 60\% employed in agriculture in 1840 to about 2.5\% today (Figure \ref{fig:share_landlabor}). The process of structural change accelerated significantly over the period 1945-1975: in 1945, 36\% of the working population are still in agriculture and this number falls below 10\% in 1975. In this sense, France is somewhat peculiar relative to the other advanced economies: it is still a largely agrarian economy right after World War II---much more than the U.K. or the U.S.

%\begin{figure}
\begin{figure}[h!]	
	\begin{center}
		\includegraphics[scale=1.0]{\modplots/../Share_EmpLand_Agr1840_FRA.pdf}
	\end{center}
	\vspace{-0.5cm}
	\caption{Land use and labor reallocation in France (1840-2015).\label{fig:share_landlabor}
	}
	{\footnotesize \textit{Notes}: The solid line shows the share of French land used for agriculture (left axis). The dashed line shows the share of workers in the agricultural sector (right axis).
		\textit{Source}: French Census for Agriculture.}
\end{figure}


%We also observe some significant cross-regional variations even though all regions exhibit structural transformation over time. At each date, some regions have larger shares of employment in agriculture, than others (e.g. in 1860 (resp. 1975), only 12\% (resp. 1\%) of people work in agriculture in the Parisian region, while at the same dates, 67\% (resp. 21\%) work in agriculture in Brittany). While the speed and timing of structural transformation is largely common across regions, the data exhibit some relevant variations. For instance, traditional manufacturing regions like Nord-Pas de-Calais saw a massive reduction of the share of agriculture in the early twentieth century, while most regions saw the drastic fall in the thirty years following World War II.

\textbf{Land use.} Although measurement is sometimes difficult for the very early periods, one can confidently argue that, in the aggregate, the share of French land used for agriculture  fell significantly since 1840 (Figure \ref{fig:share_landlabor}).\footnote{The main issue is the definition of agricultural land. Forests were part of agricultural land in the 19th century but not later. Given their use as natural amenity, we exclude them throughout, even though forest exploitation for wood production is arguably of agricultural nature. The allocation of grazing fields is also not entirely consistent across years before World War II.} Our preferred estimates are that about two thirds of French land was used for agriculture in 1840. In 2015, this number decreased to 52\%. In other words, about 15\% of French land use has been reallocated away from agriculture since 1840. While this might not seem quantitatively important, it is substantial from the perspective of urban expansion. 15\% of the French territory is actually more than the total amount of land with artificial use in France nowadays, which is about 9\% of total land. While it is difficult to assess with certainty what usage former agricultural land has been put to over such a long period, it is likely that a significant fraction of this land has been artificialized, allowing cities to expand. More precise data on land use over the period 1982-2015 show that the surface of artificialized soil increased by about 2 million hectares, or 3.7\% of the French territory. This represents roughly 70\% of the quantity of land converted away from agriculture over the same period.\footnote{Since 1982, data on land use beyond agricultural land use are available on a regular basis from the Enquêtes Teruti and Teruti-Lucas. The rest of agricultural land is to a large extent converted into forests and woods. Forests were accounting for about 18\% of French land in 1882 (Agricultural Census) compared to about 30\% in 2015 (Enquête Teruti-Lucas)---growing out of agricultural land but also rocky land, moors and sparse vegetation areas.} The measurement of cities area (presented below) provides further compelling evidence that a significant fraction of agricultural land was reallocated towards urban land use.


\subsection{Urban Expansion}\label{sec:urban-expansion}

\textbf{Data.} We use historical maps, aerial photographs and satellite data to measure the area of the main French cities at different dates: 1866 (military maps, e.g. carte d'Etat Major), 1950 (maps and/or photographs), and every ten to fifteen years after 1975 using satellite data from the Global Human Settlement Layer (GHSL) project. One caveat is that we cannot have any area measurement between 1866 and 1950. Data and procedure for the measurement of urban extent across French cities are detailed in Appendix \ref{A-sec:area-pop-measure}. Measurement of the urban extent using maps in 1866 and 1950 is performed for the 100 most populated cities in the initial period. For a given city, the urban extent ends when the land is not continuously built upon. For the satellite data, it is delimited by grid cells where the fraction of built up land is below 30\% and a requirement that cells are connected.\footnote{For maps/photos, the urban fringe is visible by a stark color change between the built and non-built part. See Figures \ref{A-fig:Reims-EM} and \ref{A-fig:Reims-1950} for the precise measurement of the area of Reims using the 1866 and 1950 maps. For satellite data, measurement is not very sensitive to alternative built up thresholds (Appendix \ref{A-sec:cutoff-sens}). Figures \ref{A-fig:built-marseille} and \ref{A-fig:built-bordeaux} illustrate how GHSL data are used to delineate the urban boundaries of Marseille and Bordeaux. We double-check the quality of photo/map measurement in the recent period relative to satellite data measurement. The cross-sectional correlation between both measures is very high. We also cross check our measures with \cite{angel2010persistent} for Paris and find very similar results. While measurement error when delineating the urban area is unavoidable at the city level, it is less of an issue when averaging across the 100 cities.} By way of example, Figure \ref{fig:reims1866} shows the 1866 map for a medium-size French city, Reims---where one can observe the sharp discontinuity of urban built at the boundary used to delimitate the urban area, even though measurement error at the city-level remains unavoidable (with some farmland included or detached houses inappropriately excluded). On the same scale,  Figure \ref{fig:reims2015} shows the same city in 2016 viewed from the sky, with an area of about 20 times larger than in 1866. This figure also clearly shows how the city is surrounded by agricultural land---a crucial element for our story where urban land expands out of farmland. This feature is not specific to Reims. Recent satellite observations from the Corine Land Cover project show that our sample of cities is surrounded mainly by agricultural land: apart from their coastal part and water bodies, two thirds of land use in the near surroundings of cities is agricultural.\footnote{The rest is made of forest/moors and discontinuous urban land (e.g. leisure/transport infrastructure, industrial/commercial sites, ...)---both categories in roughly equal proportions. See Appendix \ref{A-sec:CLC} for details.}

\begin{figure}
	\begin{subfigure}{0.5\textwidth}
		\includegraphics[width = \linewidth]{\dataplots/reims_1866.pdf}
		\caption{Etat Major map of Reims in 1866.\label{fig:reims1866}
		}
	\end{subfigure}%
	\hspace{5mm}
	\begin{subfigure}{0.5\textwidth}
		\includegraphics[width = \linewidth]{\dataplots/reims_2015.pdf}
		\caption{Photograph of Reims in 2016.\label{fig:reims2015}}
	\end{subfigure}
	\caption{The urban expansion of Reims.}
	{\footnotesize \textit{Source}: \emph{Carte d'Etat Major 1820-1866} and \emph{Photographies aériennes 2016-2020} in \url{https://www.geoportail.gouv.fr}, run by the Institut national de l’information géographique et forestière (IGN).}
\end{figure}


Using Census data, we relate the measured land area occupied by cities to the corresponding population. Data for the first available Census in 1876 are used for the initial period of study. Census data defines population at the municipality level (`commune') and an urban area can incorporate more than one municipality. In 1876, this is not a concern as the main `commune' of the city is the whole city population. In later periods, one needs to group municipalities into an urban area.  Post 1975, GHSL data combines satellite images with Census data on population. This directly provides the population of every grid cell of our measured urban area, circumventing the issue. However, for the 1950 period in between, the different municipalities that are part of our measured areas must be selected. This is done on a case by case basis, looking at the map of each of the 100 largest urban areas. This way, we make sure that the population of the area incorporates all the corresponding municipalities' population.\footnote{In 1950, only the largest cities, particularly Paris, are the result of the agglomeration of several `communes'.}  

\begin{figure}[p]	
	\begin{center}
		\includegraphics[width=0.8\textwidth]{\dataplots/figure2.pdf}
	\end{center}
	\vspace{-0.5cm}
	\caption{Urban area and population of the 100 largest cities in France (1870-2015).\label{fig:areapop}
	}
	{\footnotesize \textit{Notes}: The dashed line shows the total urban area of the 100 cities relative to the initial period (sum of all the urban areas) . The bottom solid line shows the total population relative to the initial period in the same cities. Both area and population are normalized to unity in the initial period. 
		\textit{Source}:  Etat major, IGN, GHSL and Census.}%See Appendix \ref{A-sec:area-pop-measure}.}
\end{figure}


\begin{figure}[p]
	\begin{subfigure}{0.5\textwidth}
		\includegraphics[width = \linewidth]{\dataplots/figure3a.pdf}
		\caption{The decline in average urban density.\label{fig:avdensity}
		}
	\end{subfigure}%
	\hspace{5mm}
	\begin{subfigure}{0.5\textwidth}
		\includegraphics[width = \linewidth]{\dataplots/figure3b.pdf}
		\caption{The decline in average and central density in Paris.\label{fig:avdensity-paris}}
	\end{subfigure}
	\caption{The historical decline in urban density.}
	{\footnotesize \textit{Notes}: Left panel: the solid line shows the urban density averaged across the top 100 French cities (weighted average with 1975 population weights). Right panel: the solid line shows the average urban density in Paris; the dashed line the density in Central Paris (districts 1 to 6).
		\textit{Source}: Etat major, IGN, GHSL and Census.}
\end{figure}


\textbf{The area and population of French cities over time.} Over time, cities have been increasing much faster in area than in population. Let us give some order of magnitude and describe the average evolution over time for the 100 most populated French cities in 1876. Figure \ref{fig:areapop} shows the evolution of total area and population of these 100 cities over the period considered---both variables being normalized to 1 to show the increase in size. Since 1870, the area of cities has been multiplied by a factor close to 30 on average. This is a substantial increase. Between 1870 and 1950,  the area of cities was roughly multiplied by a factor of 6. Between 1950 and today, the area of cities was multiplied again by a factor of 5 on average---the fastest rate of increase being observed over the period 1950-1975. For comparison, the population of these cities has been multiplied by a factor close to 4 since 1870.\footnote{French population was multiplied by a bit less than 2 over the entire period. Due to the reallocation of people way from rural areas towards cities, we get roughly a factor 4 over the period.} As urban area increased at a much faster rate than urban population, the average urban density significantly declined over the period.

\textbf{The density of French cities over time.} Using population and area of cities at the different dates, one can measure the evolution of urban densities across the different cities over 150 years. While in the cross-section larger cities are denser, the density of French cities declined over time---area expanding at a faster rate than population. This is shown in Figure \ref{fig:avdensity} for the population-weighted average of density across the 100 largest French cities. The average urban density fell massively over the period: it has been divided by a factor of roughly 8. Urban density fell at the fastest rate over the period 1950-1975 and barely falls thereafter. Thus, urban density fell the most over the period when people massively left rural areas and the employment share in agriculture fell the most. The later slowdown of the decline in density coincides with the slowdown in the rate of structural change.%\footnote{The historical decline in urban density is observed across all cities although the magnitude differs across cities. See Appendix \ref{A-sec:density-results} for further insights on the evolution of urban density across different cities.}

%\footnote{The historical decline in urban density is observed across all cities although the magnitude differs across cities: for instance, in Lille, urban density fell from 67,300 to about 4,200 people/km$^{2}$---divided by a factor 15 between 1866 and 2015; over the same period, urban density was divided by less than 4 in Nancy, from 13,400 to 3,500 people/km$^{2}$.}


Ideally, one would like to explore how density evolved in different locations of a city (within-city variations). This would provide information on whether density fell in the central locations or in the outskirts of the city. Unfortunately, for most cities we are not able to differentiate the central density to the suburban one as most cities expand the area of their main historical `commune', particularly so over the period 1870-1950. Thus, we cannot measure the historical population in different parts of a city. However, it can be done for Paris which is divided into several districts. Figure \ref{fig:avdensity-paris} shows the evolution of the density of Central Paris relative to the average urban density of the metropolitan area: the central density of Paris did fall over time but significantly less than the average density of the city. This suggests that the decline in average urban density is not only due to a reallocation of urban residents away from dense centers but also due to the addition of less and less dense suburban areas at the city fringe over time.



%There are also some interesting variations in terms of urban expansion across cities (or across the main cities of each of the regions). Hopefully we will be able to link these variations to differences in the reallocation of land use across regions.











%This could be interesting to have such measure to differentiate the effects driven by the reallocation of land use due to productivity growth from the effects driven by falling commuting costs. 

% \subsection{Food consumption share.} I did compute an historical serie for the food and drinks consumption share (1890 until today). This is not ideal as it includes quite a bit of industrial transformation and distribution services (the food and drinks consumption share is much higher than the share of agriculture in value added).

%  Still it can be useful as a motivation. Interestingly, the share of spending on food and drinks was very high in 1950, about 47\% (and not much below the share in 1890). This significantly fell over the period 1950-1980, being divided by about 2. One needs to be careful with this measure though as it could be that the share of value added embedded in these goods and not tied to agriculture is rising over time---in this case one would generally underestimate the fall of the share of sepnding dedicated to agricultural products over a long time period.


\begin{figure}[h!]
	\begin{subfigure}{0.5\textwidth}
		% \includegraphics[width = \linewidth]{figurecrosssection_sec2_price.pdf}
		\includegraphics[width = \linewidth]{\dataplots/figure4-a.pdf}
		\caption{Urban Density and Farmland Price.\label{fig:cross-section-density-price}}
	\end{subfigure}%
	\hspace{5mm}
	\begin{subfigure}{0.5\textwidth}
		% \includegraphics[width = \linewidth]{figurecrosssection_sec2_pop.pdf}
		\includegraphics[width = \linewidth]{\dataplots/figure4-b.pdf}
		\caption{Urban Density and Population.\label{fig:cross-section-density-pop}}
	\end{subfigure}
	\caption{Urban density across French cities in 2000.}
	{\footnotesize \textit{Notes}: Binned scatter plots. Left-panel: urban density is averaged within each decile bin of farmland prices. Right panel: urban density is averaged within each decile bin of urban population. Urban area, population and density from GHSL data, local farmland prices from the Ministry of Agriculture.}
\end{figure}


\textbf{Urban density across French cities in recent times.} Using satellite data available in the recent period, we build a larger sample of 200 cities for which we measure population and area in the recent years---adding the largest cities in population in 1975 that are not in the initial sample. While the primary focus is to describe the evolution of urban density over long period, we provide insights on a novel determinant of urban density in the cross-section: the price of farmland at the urban fringe. To do so, we use data from the Ministry of Agriculture on the average market transaction prices of arable land (per ha) (`Prix des terres agricoles, terres labourables, libres) at the level of a `Petite Région Agricole (PRA)'---with more than 700 PRAs in France, this provides a fairly local farmland price surrounding each city (see details in Appendix \ref{A-sec:spatialagri}). Averaging density across cities within each decile bin of farmland prices in 2000, the binned scatter plot in Figure \ref{fig:cross-section-density-price} shows that urban density is significantly higher in cities surrounded by more expensive, arguably more productive, farmland.\footnote{Results are similar in 2015.} Despite possible endogeneity issues treated in Section \ref{sec:rescross}, this preliminary evidence suggests an important novel fact at the heart of our story: a lower opportunity cost of expanding cities at their fringe increases urban sprawl. For comparison, we also show the link between urban population and density---more populated cities being denser (see binned scatter plot of Figure \ref{fig:cross-section-density-pop}, where urban density in averaged within each decile bin of population in 2000). This suggests a quantitatively meaningful effect of farmland prices on urban density: increasing farmland prices around cities from the first to the last decile corresponds to a density increase by about a third, an effect similar in magnitude to an increase in urban population from  about 25,000 (3rd decile) to 150,000 (9th decile). Lastly, note that this latter well-known fact linking urban population and density stands in contrast with the decline of urban density in the time-series when cities are getting larger. 



\subsection{Land values}

%\textit{This data section is not very advanced}.

\textbf{Data.} Data on land and housing values (over income) for France over a long period can be found in \cite{piketty2014capital}. Historical data for the real housing price index for France are provided in \cite{knoll2017no}. 
%\footnote{Using various data sources, we also computed a measure of farmland prices per unit of land. Our estimates are consistent with \cite{piketty2014capital}.}




%\begin{figure}[hbt!]
%	\begin{center}
%		\includegraphics[scale=0.95]{PikettyWealth_FRA1820.pdf}
%	\end{center}
%	\vspace{-0.5cm}
%	\caption{Agricultural Land and Housing Wealth (1820-2010)}
%	\label{fig:pikettywealth}
%	\footnotesize {\textit{Source:} \cite{piketty2014capital}.}
%\end{figure}

%\begin{figure}[hbt!]
%	\begin{center}
%		\includegraphics[scale=0.95]{hpreal100_FRA.pdf}
%	\end{center}
%	\vspace{-0.5cm}
%	\caption{Real Housing Price Index in France (1870-2010).}
%	\label{fig:real-hpi-knoll}
%	\footnotesize {\textit{Source:} \cite{knoll2017no}.} 
%\end{figure}




% \begin{figure}[H]
% 	\advance\leftskip-1.5cm
% 	\advance\rightskip-1.5cm
% 	\subfloat[Agricultural Land and Housing Wealth (1820-2010).\label{fig:pikettywealth}]{\begin{centering}
% 			\includegraphics[scale=0.7]{PikettyWealth_FRA1820.pdf}\par\end{centering}		
% 	}
% 	\subfloat[Real Housing Price Index in France (1870-2010).\label{fig:real-hpi-knoll}]{\begin{centering}
% 			\includegraphics[scale=0.7]{hpreal100_FRA.pdf}
% 			\par\end{centering}

% 	}
% 	%\vspace{-0.5cm}
% 	\caption{Land and Housing Value.}
% 	\advance\leftskip 1.5cm
% 	\advance\rightskip 1.5cm
% 	{\footnotesize \textit{Notes}: The left plot shows agricultural wealth as a share of national income in \% (dashed) and the sum of agricultural and housing wealth as a share of national income in \% (solid). The right plot shows the housing price index deflated by the CPI. Data are from \cite{piketty2014capital} (left plot) and \cite{knoll2017no} (right plot).}
% \end{figure}

\textbf{Historical evolution.} Figure \ref{fig:pikettywealth} describes the evolution of the aggregate value of French land over income since 1820. The fall in the value of housing and land wealth (as a share of income) in the pre-World War II period is essentially driven by a declining value of farmland. While farmland was expensive relative to income in the nineteenth century, today it is relatively cheap. This is confirmed by data on average farmland prices: since 1850, the average value of an agricultural field (per unit of land) as a share of per capita income has been divided by a factor of 15 in France. This fact is at the heart of our story: structural change puts downward pressure on farmland values---allowing cities to expand at a fast rate. As a consequence, there is an important reallocation of land values across usage, from agricultural land towards housing (or urban) land. While the value of agricultural land accounted for more than 70\% of housing and land wealth in 1820, it accounts for only 3\% in 2010. Lastly, despite the falling value of farmland as a share of income, the total value of housing and land wealth (as a share of income) grows at an increasing rate after 1950. 



This steep increase, arguably driven by the increasing value of urban land where most of the population is concentrated, echoes the findings of \cite{knoll2017no}.\footnote{\cite{bonnet2019} show that this increase in the price of housing is largely driven by the price of land and not by the capital and structure component.} They show that for developed countries, including France, housing prices have been quite stable until the 1950s before rising at an increasing pace---a \emph{hockey-stick} shape of housing prices as shown in Figure \ref{fig:real-hpi-knoll}.



\begin{figure}[h]
	\begin{subfigure}{0.5\textwidth}
		\includegraphics[width = \linewidth]{\modplots/../PikettyWealth_FRA1820.pdf}
		\caption{Agricultural Land and Housing Wealth.\label{fig:pikettywealth}}
	\end{subfigure}%
	\hspace{5mm}
	\begin{subfigure}{0.5\textwidth}
		\includegraphics[width = \linewidth]{\modplots/../hpreal100_FRA.pdf}
		\caption{Real House Price Index.\label{fig:real-hpi-knoll}}
	\end{subfigure}
	\caption{Land and Housing Values in France.}
	{\footnotesize \textit{Notes}: The left plot shows agricultural wealth as a share of French national income in \% (dashed) and the sum of agricultural and housing wealth as a share of national income in \% (solid). The right plot shows the housing price index deflated by the CPI. Data are from \cite{piketty2014capital} (panel \ref{fig:pikettywealth}) and \cite{knoll2017no} (panel \ref{fig:real-hpi-knoll}).}
\end{figure}


%A Similar evolution holds for Paris over long period. See \cite{friggit2008}. 

%Of course, one would like to have more disaggregated data (farmland versus housing/urban land) and cross-regional (and cross-cities) variations. In the recent period, we have transaction data for housing and farmland at the parcel level that we are currently collecting. These data are very geolocalized but useful only for variations across locations (almost no time-series). Further back in time, the Ministry of Agriculture provides price indices at the local level for farmland (across different types of land). Before WWII, the best we can potentially get are measures of the total value of French farmland at different dates---maybe at a regional level (or for some regions), but it seems rather difficult.





%\begin{figure}[H]





%Data on the price per ha suggests that as a share of income, the price of farmalnd ha sbeen divided by about 15.


To sum-up, our historical data shows a set of salient facts over the last 180 years: beyond the well-known reallocation of labor away from agriculture, land has been reallocated away from agricultural use. Migrations away from the rural areas were accompanied with urban expansion both in area and population. However, given that urban area grew at a significantly faster pace than urban population, the average urban density massively declined over the period, particularly so in the decades following World War II. Together with this process of structural change, the value of farmland as a share of income shrank significantly to the benefit of non-agricultural (urban) land. 

These stylized facts motivate our subsequent theoretical analysis. We introduce a spatial dimension together with endogenous land use to the standard theory of structural change with non-homothetic preferences to jointly study agricultural decline, urban sprawl and the spatial reallocation of land values.


\end{document}